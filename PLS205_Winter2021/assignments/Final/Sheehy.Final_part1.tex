% Options for packages loaded elsewhere
\PassOptionsToPackage{unicode}{hyperref}
\PassOptionsToPackage{hyphens}{url}
%
\documentclass[
]{article}
\usepackage{lmodern}
\usepackage{amssymb,amsmath}
\usepackage{ifxetex,ifluatex}
\ifnum 0\ifxetex 1\fi\ifluatex 1\fi=0 % if pdftex
  \usepackage[T1]{fontenc}
  \usepackage[utf8]{inputenc}
  \usepackage{textcomp} % provide euro and other symbols
\else % if luatex or xetex
  \usepackage{unicode-math}
  \defaultfontfeatures{Scale=MatchLowercase}
  \defaultfontfeatures[\rmfamily]{Ligatures=TeX,Scale=1}
\fi
% Use upquote if available, for straight quotes in verbatim environments
\IfFileExists{upquote.sty}{\usepackage{upquote}}{}
\IfFileExists{microtype.sty}{% use microtype if available
  \usepackage[]{microtype}
  \UseMicrotypeSet[protrusion]{basicmath} % disable protrusion for tt fonts
}{}
\makeatletter
\@ifundefined{KOMAClassName}{% if non-KOMA class
  \IfFileExists{parskip.sty}{%
    \usepackage{parskip}
  }{% else
    \setlength{\parindent}{0pt}
    \setlength{\parskip}{6pt plus 2pt minus 1pt}}
}{% if KOMA class
  \KOMAoptions{parskip=half}}
\makeatother
\usepackage{xcolor}
\IfFileExists{xurl.sty}{\usepackage{xurl}}{} % add URL line breaks if available
\IfFileExists{bookmark.sty}{\usepackage{bookmark}}{\usepackage{hyperref}}
\hypersetup{
  pdftitle={Know Your Audience, Choose Your Stage},
  hidelinks,
  pdfcreator={LaTeX via pandoc}}
\urlstyle{same} % disable monospaced font for URLs
\usepackage[margin=1in]{geometry}
\usepackage{longtable,booktabs}
% Correct order of tables after \paragraph or \subparagraph
\usepackage{etoolbox}
\makeatletter
\patchcmd\longtable{\par}{\if@noskipsec\mbox{}\fi\par}{}{}
\makeatother
% Allow footnotes in longtable head/foot
\IfFileExists{footnotehyper.sty}{\usepackage{footnotehyper}}{\usepackage{footnote}}
\makesavenoteenv{longtable}
\usepackage{graphicx,grffile}
\makeatletter
\def\maxwidth{\ifdim\Gin@nat@width>\linewidth\linewidth\else\Gin@nat@width\fi}
\def\maxheight{\ifdim\Gin@nat@height>\textheight\textheight\else\Gin@nat@height\fi}
\makeatother
% Scale images if necessary, so that they will not overflow the page
% margins by default, and it is still possible to overwrite the defaults
% using explicit options in \includegraphics[width, height, ...]{}
\setkeys{Gin}{width=\maxwidth,height=\maxheight,keepaspectratio}
% Set default figure placement to htbp
\makeatletter
\def\fps@figure{htbp}
\makeatother
\setlength{\emergencystretch}{3em} % prevent overfull lines
\providecommand{\tightlist}{%
  \setlength{\itemsep}{0pt}\setlength{\parskip}{0pt}}
\setcounter{secnumdepth}{-\maxdimen} % remove section numbering

\title{Know Your Audience, Choose Your Stage}
\author{}
\date{\vspace{-2.5em}}

\begin{document}
\maketitle

\hypertarget{experimental-overview}{%
\subsection{1. Experimental Overview}\label{experimental-overview}}

Animal behavior is often modified by the animal's environment. This
includes both the physical environment (substrate, spatial complexity,
etc.) and the social environment, such as mates, family groups, or
unrelated members of a herd. In the case of dominance interactions
(fights for social status) the familiarity of an audience may influence
how much an animal may want to invest in a fight. In a spatially complex
environment, certain locations may act as a ``stage,'' allowing the
animal to showcase it's fighting ability to an audience. Conversely,
spatial complexity may inhibit the audience's ability to see an
interaction, allowing the animal to conceal it's fighting ability. The
effects of both the physical and social environment on behavior have
been well studied in isolation, but rarely in combination.

I propose to examine the interaction of spatial complexity and social
environment on dominance interactions in the Amazon molly, a naturally
genetically clonal (and all female) species of fish. In addition to
being clones, these fish are a useful study system to study dominance
interactions as fish of the same size will reliably engage in fights for
dominance.

Fish will be placed in groups of 8 size-matched fish and allowed to
become familiar over 3 weeks. After this, a pair will be randomly
selected and placed into separate acclimation chambers within a
Dominance Interaction tank. This tank will either have a covering which
obscures 1/2 of the tank (Complex) or be completely open and visible
(Simple). The remainder of the group OR an unfamiliar group of 6 fish
(of similar size) will be placed in an adjacent Audience Tank, where
they will be allowed to observe the pair in the Dominance Interaction
tank. Fish are readily able to view each other tank to tank (except when
the barrier is in place, which obscures 1/2 of the view).

After the acclimation period, the dominance pair will be released from
their chambers and allowed to fight for 5 minutes. Aggressive acts
(bites, tail beats, and chases) will be recorded and combined to create
a single ``Aggression'' score for the pair's interaction.

\hypertarget{design-table}{%
\subsection{2. Design Table}\label{design-table}}

\begin{longtable}[]{@{}lllll@{}}
\toprule
Structure & Variable & Type & \#levels & EU\tabularnewline
\midrule
\endhead
Treatment & Audience & Cat & 2 & Group:Aud\tabularnewline
& Tank Complexity & Cat & 2 & Group:Comp\tabularnewline
& Aud:Comp & Cat & 4 & Group:Aud:Comp\tabularnewline
Design & Group:Aud & Cat & 8 &\tabularnewline
& Group:Comp & Cat & 8 &\tabularnewline
& Group:Aud:Comp & Cat & 16 &\tabularnewline
& Group & Cat & 4 &\tabularnewline
& Pair & Cat & 16 &\tabularnewline
& Pair:Aud & Cat & 16 &\tabularnewline
& Pair:Comp & Cat & 16 &\tabularnewline
& Pair:Aud:Comp & Cat & 16 &\tabularnewline
Response & Aggression & Numeric & 16 &\tabularnewline
\bottomrule
\end{longtable}

\begin{quote}
Because the treatment of familarity is only uniformly applied at the
Group level, I have chosen Group as the EU. Group is also a block, as
each ``familar audience'' treatment is necessarily different for each
Group. Pair can also be considered a block, but is nested within Group.
\end{quote}

\hypertarget{linear-model}{%
\subsection{3. Linear model}\label{linear-model}}

\begin{quote}
mollymolly\_model \textless- lmer(Aggression \textasciitilde{} Aud +
Comp + Aud:Comp + Group + (1\textbar Group:Aud) + (1\textbar Group:Comp)
+ (1\textbar Group:Aud:Comp) + Pair + (1\textbar Pair:Aud) +
(1\textbar Pair:Comp) + (1\textbar Pair:Aud:Comp))
\end{quote}

\hypertarget{effect-estimates}{%
\subsection{4. Effect Estimates}\label{effect-estimates}}

\begin{quote}
Using an ANOVA (and the emmip() function), I will compare the treatment
effect estimates of Familiar vs.~Unfamiliar, Complex vs.~Simple, and the
interaction of the two (Aud:Comp) to determine if there is evidence of
an effect of Audience, Complexity, or an interaction of the two, on
Aggression (units = mean Aggression).
\end{quote}

\begin{quote}
Example Statement: We found that there was a significant interaction
between Tank Complexity and Audience Familiarity. An unfamiliar audience
always decreased aggression relative to a familiar audience (p-value =
0.001). However, tank complexity varied depending on the audience
familiarity, with unfamiliar audiences and simple tanks leading to the
largest decrease in aggression relative to the other treatment
combinations.
\end{quote}

\hypertarget{blocking}{%
\subsection{5. Blocking}\label{blocking}}

\begin{quote}
My experiment will be a Complete Block Design, as every treatment
combination occurs in every Group. There is a possibility that the
Groups I create will have different collective behavior (and thus, Pair
behavior), so I need to have several separate Groups to control for any
variation inherent in the Group. In an ideal world, I would have more
replicates within a group, but it is unlikely that I will have enough
fish of the same size to create groups of 16 or more (2 pairs with each
treatment combo). I plan on re-assessing as I get closer to experiment
execution and determining whether an Incomplete Block Design is feasible
with the fish I have available.
\end{quote}

\end{document}
